% OBS DETTA ÄR INTE EN OFFICIELL RAPPORTMALL
% OBS DETTA ÄR INTE EN OFFICIELL RAPPORTMALL
% OBS DETTA ÄR INTE EN OFFICIELL RAPPORTMALL
\begin{center}

% Skolans loggo, ShareLaTeX fixar inte eps-filer men pdf går bra därför har jag konventerat logotypen till pdf
% som också är vectorformat.
\includegraphics[width=12.2 cm]{Bilder/JTH_cmyk_B-eps-converted-to.pdf}\centering


 \begin{center}
\textsc{\LARGE  Rapport gruppuppgift 1\\diskussion och svar}\\[1.5cm]
% Rapportens rubrik, "\\" Anger ny rad. rymms namnet på en rad låt det vara på en rad. 


\textsc{\Large Kursnamnet}\\[0.5cm]
% Kursnamnet som rapporten ingår i.
\end{center}




\begin{textblock}{0.2}[0.5,0.2](0.9,13.4)
\begin{rotate}{90}
\footnotesize Kursstart ht -03 eller senare
\end{rotate}
\end{textblock}
 % Taget från JTHs ordinarie rapportmall, vet ej varför det finns där.
 
 
\vfill
\begin{flushleft} \large
\textsc{Kontaktperson:}
Eventuell \textsc{Kontaktperson}
% Kontaktperson om sådan finns

\textsc{Författare:}
Anders \textsc{Andersson} (DIS 2)
% Författarnamn, oftast dit namn.

\large
\textsc{Handledare:} 
Lärare \textsc{Larson}, Lärare \textsc{Olsson}
%Handledare oftast laborationsassistenten 

\textsc{Datum:}
\today 
% Datum för när rapporten kompilerades. 

\end{flushleft}

\end{center}
\thispagestyle{empty}
\pagebreak

\begin{abstract}
En kortare beskrivning på rapporten.
\end{abstract}
\setcounter{page}{1}
\newpage
\tableofcontents

\newpage